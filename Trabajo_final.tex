%%%%%%%%%%%%%%%%%%%%%%%%%%%%%%%%%%%%%
% Stylish Article
% LaTeX Template
% Version 2.0 (13/4/14)
%
% This template has been downloaded from:
% http://www.LaTeXTemplates.com
%
% Original author:
% Mathias Legrand (legrand.mathias@gmail.com)
%
% License:
% CC BY-NC-SA 3.0 (http://creativecommons.org/licenses/by-nc-sa/3.0/)
%
%%%%%%%%%%%%%%%%%%%%%%%%%%%%%%%%%%%%%%%%%

%----------------------------------------------------------------------------------------
%	PACKAGES AND OTHER DOCUMENT CONFIGURATIONS
%----------------------------------------------------------------------------------------

\documentclass[fleqn,10pt]{SelfArx} % Document font size and equations flushed left



%----------------------------------------------------------------------------------------
%	COLUMNS
%----------------------------------------------------------------------------------------

\setlength{\columnsep}{0.55cm} % Distance between the two columns of text
\setlength{\fboxrule}{0.75pt} % Width of the border around the abstract

%----------------------------------------------------------------------------------------
%	COLORS
%----------------------------------------------------------------------------------------

\definecolor{color1}{RGB}{0,0,90} % Color of the article title and sections
\definecolor{color2}{RGB}{0,20,20} % Color of the boxes behind the abstract and headings

%----------------------------------------------------------------------------------------
%	HYPERLINKS
%----------------------------------------------------------------------------------------

\usepackage{hyperref} % Required for hyperlinks
\hypersetup{hidelinks,colorlinks,breaklinks=true,urlcolor=color2,citecolor=color1,linkcolor=color1,bookmarksopen=false,pdftitle={Title},pdfauthor={Author}}

%----------------------------------------------------------------------------------------
%	ARTICLE INFORMATION
%----------------------------------------------------------------------------------------

\JournalInfo{Taller de Biotecnología Animal, 2014-I} % Journal information
\Archive{Review} % Additional notes (e.g. copyright, DOI, review/research article)

\PaperTitle{Transfección en Animales} % Article title

\Authors{Juan Manuel Iglesias Artola\textsuperscript{1}, Gianfranco Villamonte Cuneo\textsuperscript{1}} % Authors
\affiliation{\textsuperscript{1}\textit{School of Biology, Universidad Ricardo Palma, Lima, Peru}} % Author affiliation
%\affiliation{\textsuperscript{2}\textit{Department of Chemistry, University of Examples, London, United Kingdom}} % Author affiliation
\affiliation{*\textbf{Corresponding authors}: jmanuel9112@icloud.com / giancuneo@gmail.com } % Corresponding author


%----------------------------------------------------------------------------------------
%	ABSTRACT
%----------------------------------------------------------------------------------------

\Abstract{}


\Keywords{Keyword1 --- Keyword2 --- Keyword3} % Keywords - if you don't want any simply remove all the text between the curly brackets
\newcommand{\keywordname}{Palabras clave} % Defines the keywords heading name

%----------------------------------------------------------------------------------------

\begin{document}

\flushbottom % Makes all text pages the same height

\maketitle % Print the title and abstract box

\tableofcontents % Print the contents section

\thispagestyle{empty} % Removes page numbering from the first page

%----------------------------------------------------------------------------------------
%	ARTICLE CONTENTS
%----------------------------------------------------------------------------------------

\section{Introducción} 

\addcontentsline{toc}{section}{Introducción} % Adds this section to the table of contents


 and some mathematics $\cos\pi=-1$ and $\alpha$ in the text\footnote{And some mathematics $\cos\pi=-1$ and $\alpha$ in the text.}.

%------------------------------------------------

\section{Métodos de Transfección}

\begin{figure*}[ht]\centering % Using \begin{figure*} makes the figure take up the entire width of the page
\includegraphics[width=\linewidth]{images/view}
\caption{Wide Picture}
\label{fig:view}
\end{figure*}


\begin{equation}
\cos^3 \theta =\frac{1}{4}\cos\theta+\frac{3}{4}\cos 3\theta
\label{eq:refname2}
\end{equation}



\begin{enumerate}[noitemsep] % [noitemsep] removes whitespace between the items for a compact look
\item First item in a list
\item Second item in a list
\item Third item in a list
\end{enumerate}

\subsection{Subsection}



\paragraph{Paragraph} %\lipsum[7] % Dummy text
\paragraph{Paragraph} %\lipsum[8] % Dummy text

\subsection{Subsection}



\begin{figure}[ht]\centering
\includegraphics[width=\linewidth]{images/results}
\caption{In-text Picture}
\label{fig:results}
\end{figure}

Reference to Figure \ref{fig:results}.

%------------------------------------------------

\section{Aplicaciones de la Transfección}

La transfección es una herramienta para la modificación genética. Gracias a sus múltiples métodos es posible realizar innumerables aplicaciones como:
\begin{itemize}[noitemsep] % [noitemsep] removes whitespace between the items for a compact look
\item \textbf{Terapia Génica}: Cura de enfermedades.
\item \textbf{Animales Transgénicos}: Mejoramiento para alimentación.
\item \textbf{Estudios Genéticos}: Genes 'knockout' o 'knockdown' para estudiar su expresión fenotípica.
\item \textbf{Fábricas Vivientes}: Producción de proteínas recombinantes utilizando animales.
\end{itemize}


\subsection{\textbf{Terapia Génica}}

La terapia génica es la aplicación de la transfección a la medicina. Muchos de sus usos continuan en experimentación \textit{'in vitro'} o en animales, sin embargo, este campo tiene un gran potencial.
\begin{itemize}[noitemsep] % [noitemsep] removes whitespace between the items for a compact look
\item \ Cáncer
\item \ Enfermedades congénitas
\item \ Curación de heridas , quemaduras y otras afecciones epiteliales
\end{itemize}

\subsection{Animales Transgénicos}

Los animales transgénicos son aquellos en los cuales se ha modificado parte del genoma, utilizando la transfección para introducirle genes funcionales de otro organismo. Principalmente, esto se realiza con el objetivo de incrementar la producción o las cualidades nutricionales de un organismo animal destinado a alimentación.


\subsection{Estudios Genéticos}

%\lipsum[13] % Dummy text

\begin{itemize}[noitemsep] % [noitemsep] removes whitespace between the items for a compact look
\item First item in a list
\item Second item in a list
\item Third item in a list
\end{itemize}

\subsection{Fábricas Vivientes}

%\lipsum[14] % Dummy text

\subsection{Subsection}

%\lipsum[15-23] % Dummy text

%------------------------------------------------

\section{Implicancias Bioéticas de la Transfección}

%\lipsum[10] % Dummy text

%------------------------------------------------
\phantomsection
\section*{Acknowledgments} % The \section*{} command stops section numbering

\addcontentsline{toc}{section}{Acknowledgments} % Adds this section to the table of contents

So long and thanks for all the fish \cite{khaliligene2006}.

%----------------------------------------------------------------------------------------
%	REFERENCE LIST
%----------------------------------------------------------------------------------------

\newpage
\section*{Referencias Bibliográficas}
\addcontentsline{toc}{section}{Referencias Bibliográficas}
\bibliographystyle{unsrt}
\bibliography{referencias}

%----------------------------------------------------------------------------------------

\end{document}
